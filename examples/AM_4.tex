\documentclass{article}
\usepackage[margin=4cm]{geometry}
\usepackage{amsmath}

\title{Angular momentum is really fascinating}
\begin{document}
\maketitle

Angular momentum plays a major role in galaxy formation. In spiral
galaxies, it dictates the size and alignment of the disk; elliptical galaxies, by contrast, are dispersion-supported and have on average eight
times lower stellar specific angular momentum $l_\star$ at a given stellar mass (Fall 1983; Romanowsky \& Fall 2012; Fall \& Romanowsky
2018; Harrison et al. 2017; Espejo Salcedo et al. 2022). Moreover, the
angular momentum of neighbouring galaxies is partially correlated.
This effect, known as intrinsic alignment (Troxel \& Ishak 2015),
needs to be properly modelled to disentangle it from cosmic shear
and thus ensure the success of upcoming cosmological weak-lensing
surveys (Euclid, Laureijs et al. 2011; Vera Rubin Observatory, Ivezić
et al. 2019).

We perform these three simulations with hydrodynamics using
the adaptive mesh refinement (AMR) code ramses (Teyssier 2002),
adopting a minimum cell size of 35 pc, and a mass resolution of $M_\mathrm{DM} = 1.6 \times 10^6 M_\mathrm{sun}$ and $M_\star = 1.1 \times 10^4 M_\mathrm{sun}$ for DM and stars
respectively. We use the cell resolution as the gravitational softening
length for gas and stellar particles and the maximum resolution attained in the DM-only run (constant 566 pc comoving, i.e. 186 pc at
$z=2$) for DM particles (same mass resolution and softening lengths
as NewHorizon, Dubois et al. 2021 and Obelisk, Trebitsch et al.
2021). We employ Monte-Carlo tracer particles to recover the Lagrangian history of the baryons, as described in Cadiou et al. (2019).
Lagrangian particles are passive tracers that follow mass flows between cells of the AMR grid. Monte-Carlo tracer particles achieve this by moving tracers from one cell to another with a probability proportional to the mass flux between the cells. Similarly, star formation, black hole accretion and feedback processes are traced following
fluxes between gas cells, stellar particles and supermassive black hole
particles.

We scale the magnitude of the initial angular momentum with
the genetic modification technique described above, applied to the
baryonic patch at $z = 200$. We note that, due to correlations in the
initial conditions, changes to the baryonic initial angular momentum
may also affect the dark matter angular momentum. For each of the
three reference galaxies, we generate four additional galaxy initial
conditions where the angular momentum $l_0$ of the baryonic patch
has been scaled relative to the reference angular momentum $l_{0,\mathrm{ref}}$ by a
factor $j_0/ j_{0,\mathrm{ref}} = 0.66, 0.8, 1.2$ and $1.5$ respectively. Such amplitudes
are large enough to sample a wide range of stellar angular momentum,
as we will show later, while still allowing control of the angular
momentum of individual regions in the evolved universe within a
few tens of percent (Cadiou et al. 2021). We keep the mean density
of the entire dark matter halo Lagrangian patch fixed, which causes
the halo mass to be almost unchanged at $z = 2$ (to an extent that we
will shortly quantify).

\end{document}